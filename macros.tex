%%%%%% TIKZ %%%%%%%%%%%%%%%%%%%%%%%%

\tikzstyle{startend}=[rectangle, rounded corners, minimum width=2cm,  minimum height=1cm, text width =2cm, text centered, draw=none, fill= orange!50, font=\sf]
\tikzstyle{io}=[trapezium, trapezium left angle=70,trapezium right angle= 110,minimum width=3.9cm, minimum height=1cm, text centered, draw=none, fill= blue!30, font=\sf]

\tikzstyle{process}=[rectangle, minimum width=3cm,maximum width=3, minimum height=1cm, text centered, draw=black, fill= orange!30, font=\sffamily]
\tikzstyle{Vprocess}=[rectangle, minimum width=6cm, minimum height=1cm, text centered, font=\sf\bfseries,  fill= gray!80, draw=none, text=white]
\tikzstyle{VSprocess}=[rectangle, minimum width=1cm, minimum height=2cm, text centered, font=\sf\bfseries,  fill= gray!80, draw=none, text=white]
\tikzstyle{squareprocess}=[rectangle, minimum width=2cm, minimum height=7cm, text centered, draw=black, fill= orange!30, font=\sffamily]
\tikzstyle{Bprocess}=[rectangle, minimum width=6cm, minimum height=1cm, text centered,  font=\sf\bfseries,  fill= cyan!60!black, draw=none, text=white]
\tikzstyle{BSprocess}=[rectangle, minimum width=3cm, minimum height=1cm, text centered, draw=black, fill= orange!30, font=\sffamily]
\tikzstyle{decision}=[diamond, minimum width=2.5cm, minimum height=1.5cm, align=center, inner sep=-5pt ,font=\sf\bfseries,  fill= PineGreen!60, draw=none, text=white]
\tikzstyle{IO}=[text=white]

\tikzstyle{arrow}=[line width=1.5pt, ->, >=stealth, gray!80!black]
\tikzstyle{arrowcaption}=[font=\sf\relsize{+1},black]
\tikzstyle{input}=[fill= gray!80!black, inner sep=5pt,rounded corners=5pt]
\tikzstyle{output}=[fill= gray!80!black, inner sep=5pt,rounded corners=5pt]

\pgfkeys{/heat/.is family, /heat,
	Max colour/.initial = Green4,
	Min colour/.initial = Red1,
	max colour/.initial = SpringGreen3,
	mid colour/.initial = white,
	min colour/.initial = Yellow1,
	text colour/.initial = black,
	Min color/.style = {Min colour=#1},% for our friends who can't spell
	Max color/.style = {Max colour=#1},
	min color/.style = {min colour=#1},
	mid color/.style = {mid colour=#1},
	max color/.style = {max colour=#1},
	text color/.style = {text colour=#1},
	min/.initial = -1,
	mid/.initial = 0,
	max/.initial = 1,
	slider/.code={%
		\tikz{\shade[left color=\HVal{min colour},%
			right color=\HVal{max colour}]%
			(current page.south west) rectangle ++(#1,12pt);
		}%
	}%
}

\newcommand{\tikzcircle}[2][red,fill=red]{\tikz[baseline=-0.5ex]\draw[#1,radius=#2] (0,0) circle ;}%
\newcommand\Heatset[1]{\pgfkeys{/heat, #1}}
\newcommand\HVal[1]{\pgfkeysvalueof{/heat/#1}}

\newcolumntype{H}{>{\collectcell\Heat}r<{\endcollectcell}}
\newcommand\Heat[1]{% \Heat{number in the interval [min, max] }
	\if\relax\detokenize{#1}\relax% empty cell
	\else%
	\pgfmathparse{int(100*(#1-\HVal{min})/(\HVal{max}-\HVal{min}))}% map number to [0,100]
	\ifnum\pgfmathresult>100% too big
	\edef\HeatCell{\noexpand\cellcolor{\HVal{Max colour}}}%
	\else\ifnum\pgfmathresult<0% too small
	\edef\HeatCell{\noexpand\cellcolor{\HVal{Min colour}}}%
	\else\ifnum\pgfmathresult<50% between min and mid
	\pgfmathparse{int(2*\pgfmathresult)}% map number to [0,100]
	\edef\HeatCell{\noexpand\cellcolor{\HVal{mid colour}!\pgfmathresult!\HVal{min colour}}}%
	\else% between min and max
	\pgfmathparse{int(2*(\pgfmathresult-50))}% map number to [0,100]
	\edef\HeatCell{\noexpand\cellcolor{\HVal{max colour}!\pgfmathresult!\HVal{mid colour}}}%
	\fi%
	\fi%
	\fi%
	\HeatCell\textcolor{\HVal{text colour}}{$#1$}%
	\fi%
}

\pgfkeys{/heatsec/.is family, /heatsec,
	Max colour/.initial = Green4,
	Min colour/.initial = Red1,
	max colour/.initial = SpringGreen3,
	mid colour/.initial = white,
	min colour/.initial = Yellow1,
	text colour/.initial = black,
	Min color/.style = {Min colour=#1},% for our friends who can't spell
	Max color/.style = {Max colour=#1},
	min color/.style = {min colour=#1},
	mid color/.style = {mid colour=#1},
	max color/.style = {max colour=#1},
	text color/.style = {text colour=#1},
	min/.initial = -1,
	mid/.initial = 0,
	max/.initial = 1,
	slider/.code={%
		\tikz{\shade[left color=\HVal{min colour},%
			right color=\HVal{max colour}]%
			(current page.south west) rectangle ++(#1,12pt);
		}%
	}%
}
\newcommand\HeatSecset[1]{\pgfkeys{/heatsec, #1}}
\newcommand\HSVal[1]{\pgfkeysvalueof{/heatsec/#1}}

\colorlet{BadCol}{Burlywood1!70!red}


\newcolumntype{S}{>{\collectcell\HeatSec}r<{\endcollectcell}}
\newcommand\HeatSec[1]{% \Heat{number in the interval [min, max] }
	\if\relax\detokenize{#1}\relax% empty cell
	\else%
	\pgfmathparse{int(100*(#1-\HSVal{min})/(\HSVal{max}-\HSVal{min}))}% map number to [0,100]
	\ifnum\pgfmathresult>100% too big
	\edef\HeatCell{\noexpand\cellcolor{\HSVal{Max colour}}}%
	\else\ifnum\pgfmathresult<0% too small
	\edef\HeatCell{\noexpand\cellcolor{\HSVal{Min colour}}}%
	\else\ifnum\pgfmathresult<50% between min and mid
	\pgfmathparse{int(2*\pgfmathresult)}% map number to [0,100]
	\edef\HeatCell{\noexpand\cellcolor{\HSVal{mid colour}!\pgfmathresult!\HSVal{min colour}}}%
	\else% between min and max
	\pgfmathparse{int(2*(\pgfmathresult-50))}% map number to [0,100]
	\edef\HeatCell{\noexpand\cellcolor{\HSVal{max colour}!\pgfmathresult!\HSVal{mid colour}}}%
	\fi%
	\fi%
	\fi%
	\HeatCell\textcolor{\HSVal{text colour}}{$#1$}%
	\fi%
}

%%%%%% MACROS %%%%%%%%%%%%%%%%%%%%%%%%

\definecolor{lightsalmon}{rgb}{1.0, 0.63, 0.48}
\definecolor{lightseagreen}{rgb}{0.13, 0.7, 0.67}
\definecolor{americanrose}{rgb}{1.0, 0.01, 0.24}


\DeclareMathOperator*{\argmin}{\arg\!\min}
\DeclareMathOperator*{\argmax}{\arg\!\max}
\newcommand{\multicoomment}[1]{}
\newcommand{\Software}[1]{\text{\ttfamily\bfseries #1}}
\newcommand{\OurTool}{\Software{IPANEMAP}\xspace}
\newcommand{\SM }{{\tt SHAPEMap}\xspace}
\newcommand{\SH }{{\tt SHAPE}\xspace}
\newcommand{\VP }{{\tt Vienna package}\xspace}
\newcommand{\OurRna}{\Software{Did}\xspace}
\newcommand{\mm }{{\tt$M\&M$}\xspace}
\newcommand{\DP }{{\tt DP}\xspace}
\newcommand{\didy }{DiLCrz\xspace}

\newcommand{\CE }{{\tt capillary electrophoresis}\xspace}
%MPCRnas MultiProbing Conformers}}
% Macros for # variables
\newcommand{\BP }{{\mathcal{ BP}}}
\newcommand{\Ensemble }{{\mathcal{ S}}}
\newcommand{\Sample }{{\mathcal{ S_D}}}
\newcommand{\PData }[1]{{\mathcal{ D}_{#1}}}
\newcommand{\Bzcond}[1]{ \mathbb{P}(s\mid #1)}
\newcommand{\CBP}[1]{ \mathbb{CP}_#1}
\newcommand{\BF}{ \mathbb{BF}}
\newcommand{\Zed}{\mathbb{Z}}
\newcommand{\Edist }{{ \text{Dist}}}
\newcommand{\RL }{{n}}
\newcommand{\CL}{MBkM\xspace}
\newcommand{\Clusters}{\mathcal{C}}
\newcommand{\Centroids}{\mathcal{C_O}}
\newcommand{\GMean}{\text{GM}}
\newcommand{\Ref}{\RevA{S^{\star}}}
%\newcommand{\OurRna}{\Software{Did}}
%MPCRnas MultiProbing Conformers}}
\newcommand{\NumClust}{k}
\newcommand{\NumStruct}{M}
\newcommand{\etal}{~et al. }
\newcommand{\Def}[1]{{\em #1}}

\newcommand{\Ab}{\text{\sffamily A}}
\newcommand{\Gb}{\text{\sffamily G}}
\newcommand{\Cb}{\text{\sffamily C}}
\newcommand{\Ub}{\text{\sffamily U}}


%%% Conditions
\newcommand{\Cond}[5]{\textsc{#1-#3$^{\text{#2}}_{\text{#4}}$#5}}

\newcommand{\OneMSevILUMg}{\Cond{1M7}{mg}{MaP}{il}{}\xspace}
\newcommand{\OneMSevILU}{\Cond{1M7}{}{MaP}{il}{}\xspace}

\newcommand{\OneMSevILUThreeMg}{\Cond{1M7}{mg}{MaP}{il}{-3d}\xspace}
\newcommand{\OneMSevILUThree}{\Cond{1M7}{}{MaP}{il}{-3d}\xspace}

\newcommand{\OneMSevMgCE}{\Cond{1M7}{mg}{CE}{}{}\xspace}
\newcommand{\OneMSevCE}{\Cond{1M7}{}{CE}{}{}\xspace}

\newcommand{\CMCTMg}{\Cond{CMCT}{mg}{CE}{}{}\xspace}

\newcommand{\NMIA}{\Cond{NMIA}{}{MaP}{it}{}\xspace}
\newcommand{\NMIAMg}{\Cond{NMIA}{mg}{MaP}{it}{}\xspace}

\newcommand{\NMIACE}{\Cond{NMIA}{}{CE}{}{}\xspace}
\newcommand{\NMIAMgCE}{\Cond{NMIA}{mg}{CE}{}{}\xspace}

\newcommand{\NAIMg}{\Cond{NAI}{mg}{CE}{}{}\xspace}
\newcommand{\NAICE}{\Cond{NAI}{}{CE}{}{}\xspace}

\newcommand{\BzCN}{\Cond{BzCN}{}{CE}{}{}\xspace}
\newcommand{\BzCNMg}{\Cond{BzCN}{mg}{CE}{}{}\xspace}

\newcommand{\DMSMg}{\Cond{DMS}{mg}{CE}{}{}\xspace}

\newcommand{\BZCNCE}{\Cond{BzCN}{}{CE}{}{}\xspace}


\newcommand{\Mut}{{\sc MaP}}
\newcommand{\Stop}{\sc CE}
\newcommand{\Mg}{$\bullet$}
\newcommand{\NoMg}{$\circ$}



\newcommand{\Draft}[1]{{#1}}
\newcommand{\bs}[1]{\Draft{\todo[color=red!30]{\sf Bruno: #1}}}
\newcommand{\bsi}[1]{\Draft{\todo[color=red!30,inline]{\sf Bruno: #1}}}
\newcommand{\as}[1]{\Draft{\todo[color=green!70!black]{\sf Afaf: #1}}}
\newcommand{\yp}[1]{\Draft{\todo[color=blue!30]{\sf Yann: #1}}}
\newcommand{\ypi}[1]{\Draft{\todo[color=blue!30,inline]{\sf Yann: #1}}}

%\renewcommand{\bsi}[1]{}
%\renewcommand{\ypi}[1]{}

\newcommand{\ipanemapurl}{https://github.com/afafbioinfo/IPANEMAP}

\newcommand{\Bull}[1]{{\sffamily #1}~\raisebox{1pt}{\tikzcircle[black, fill=cluster#1]{3pt}}}

\newcommand{\BullLab}[1]{\Bull{#1}}

\colorlet{clusterA}{SeaGreen}
\colorlet{clusterB}{Yellow}
\colorlet{clusterC}{gray}
\definecolor{clusterD}{HTML}{AFAFE9}
\colorlet{clusterE}{OliveGreen}
\colorlet{clusterF}{blue!90!black}
\colorlet{clusterG}{Orange}
\colorlet{clusterH}{SeaGreen!40}

\newcommand{\RevA}[1]{{#1}}
\newcommand{\RevB}[1]{{\color{red!40!black!80!blue} #1}}